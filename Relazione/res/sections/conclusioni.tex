\section{Conclusioni}

In questa sezione esporrò alcune delle mie considerazioni finali sul tirocinio.

\subsection{Metodologie e workflow}
Durante il tirocinio ho potuto testare il telelavoro full-time. Da una parte credo che mi abbia penalizzato, dall'altra invece aiutato parecchio. A parte il periodo iniziale di formazione in ufficio, la maggior parte delle ore lavorative le ho passate a casa. La penalizzazione viene dalle poche relazioni strette con i colleghi. Ho avuto modo di conoscere poche persone e non sempre non mi è stato chiaro il ruolo di tutti o la struttura della gerarchia aziendale. Per forze di causa maggiore non è stato possibile lavorare in ufficio, ma questo è stato sicuramente un fattore di comodità, poter utilizzare il mio hardware e guadagnare tutto il tempo dovuto allo spostamento. \\
Dal punto di vista di organizzazione del lavoro e dei team ho avuto il piacere di collaborare con una azienda seria, che si preoccupa di creare un ambiente lavorativo produttivo. \\
Per quanto riguarda le metodologie, ho notato che vengono utilizzate le più comuni e moderne, come alcuni concetti di Agile e SCRUM, ma non sono ufficializzate in nessun documento o insieme di norme. 
Tuttavia ho potuto toccare anche con mano un'insieme di best practice studiate, ma mai applicate. 

\subsection{Tecnologie utilizzate}
Un aspetto che mi ha leggermente deluso sono state le tecnologie utilizzate durante il tirocinio, spesso datate o atipiche. Mi sono presto ricreduto, dato che mi sono reso conto di quante risorse siano necessarie a un aggiornamento su infrastrutture grandi e controllate come quella di \textit{Sportello}. Ad esempio, utilizzare un IDE diverso più aggiornato di Visual Studio 2013 probabilmente richiederebbe dei cambiamenti, anche a livello di infrastruttura, che i benefici non basterebbero a pareggiare con le risorse utilizzate. Questo esempio si può estendere alle librerie, che richiderebbero mesi per essere sostituite per ottenere miglioramenti minimi. \\ Ho visto che comunque rimane un mondo in continuo cambiamento e in aggiornamento, che cerca sempre di inseguire la miglior configurazione possibile. Sempre secondo la filosofia: "se funziona quanto basta, non toccare che si guasta", imparata all'università.

\subsection{Conoscenze acquisite}
Oltre ad aver utilizzato un gran numero di tecnologie e strumenti, da me ancora inesplorati, ho imparato a lavorare in un organizzazione strutturata. La divisione in team e la collaborazione, erano esperienze che non avevo ancora provato, nemmeno in altri stage precedenti. Ho avuto modo anche di elaborare un metodo di lavoro adatto ad approciarmi a grossi sistemi, con un grado di complessità elevato.